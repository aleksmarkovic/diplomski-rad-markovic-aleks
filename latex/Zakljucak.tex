% !TeX encoding = windows-1250
\chapter{Zaklju�ak}

\qquad U ovom se radu razvila aplikacija za upravljanje robotom i mapiranje prostora koriste�i SLAM gmapping. Paralelno s upravljanjem robota generira se 2D mapa prostora, nalik tlocrtu zgrade. Tako�er, omogu�eno je generiranje sfera iz podataka laserskog skena, kao i prikaz iz prvog lica kamere robota. U slu�aju vi�e robota u simulaciji, mogu�a je promjena perspektive s jedne kamere robota na kameru drugog robota.

Aplikacija je ra�ena isklju�ivo koriste�i Turtlebot3 simulaciju. Za adaptaciju na drugu vrstu robota trebalo bi biti dovoljno u�itati novi URDF model te eventualno izmijeniti ROS teme u \emph{RosConnector-u} u slu�aju da se razlikuju.

Aplikacija je izgra�ena i testirana na tri platforme: Windows, Linux Ubuntu i Android. Mogu�e je uz minimalne promjene aplikaciju izgraditi i za Mac OS. Performanse aplikacija nisu savr�ene kao �to je navedeno u zadnjem poglavlju rada, no aplikacija za Windows je drasti�no lo�ija od ostalih iako je sve prakti�ki isti kod.

ROS\# se pokazao kao alat s vrlo velikim potencijalom za izradu konkretnijih aplikacija, kao i za edukacijske svrhe kao te za entuzijaste koji rade s Unity razvojnim okru�enjem.

Zaklju�no, glavni cilj ovog rada - spajanje robota na Unity i implementiranje multiplatformske aplikacije, dodatno podi�e mogu�nosti ROS-a i Unity-ja koji zajedno mogu dovesti do vrlo zanimljivih i korisnih rezultata.