% !TeX encoding = windows-1250
\chapter{Zaklju�ak}

\qquad U ovom se radu razvila aplikacija za upravljanje robotom i kartiranje prostora koriste�i \emph{SLAM gmapping} za 2D kartu te \emph{Octomap mapping} za 3D kartu. Paralelno s upravljanjem robota generira se 2D karta prostora, nalik tlocrtu zgrade. Tako�er, omogu�eno je generiranje sfera iz podataka laserskog skena, kao i prikaz iz prvog lica kamere robota. U slu�aju vi�e robota u simulaciji, mogu�a je promjena perspektive s jedne kamere robota na kameru drugog robota. Napravljeno je i generiranje 3D karte prostora u stvarnom vremenu koje se odra�uje u zasebnoj sceni.

Aplikacija je ra�ena isklju�ivo koriste�i Turtlebot 3 - Waffle simuliranog robota. Za adaptaciju na drugu vrstu robota dovoljno je u�itati novi URDF model te eventualno izmijeniti ROS teme u \emph{RosConnectoru} u slu�aju da se razlikuju.

Aplikacija je izgra�ena i testirana na tri platforme: Windows, Linux Ubuntu i Android. Mogu�e je uz minimalne promjene aplikaciju izgraditi i za Mac OS. Performanse aplikacija nisu savr�ene kao �to je navedeno u zadnjem poglavlju rada, gdje iste ovise o fazi rada i kori�tenom hardveru. U fazi 1 gdje je kori�tena \emph{Waffle Pi} ina�ica Turtlebota aplikacija za Windows je drasti�no lo�ija od ostalih iako je sve prakti�ki isti kod. Naime, u fazi 2 je kori�tena \emph{Waffle} ina�ica Turtlebota s kamerom koja omogu�ava 3D percepciju prostora gdje je fluidnost slike kamere drasti�no pala. Me�utim, scena koja generira 3D kartu sa�uvala je logi�nost izme�u kori�tenog hardvera i performansa aplikacije gdje su ra�unalne verzije odr�ale normalan i konstantan broj a�uriranja slike u sekundi (FPS), a mobilna se verzija uvelike usporila.

ROS\# se pokazao kao alat s vrlo velikim potencijalom za izradu konkretnijih aplikacija, kao i za edukacijske svrhe te za entuzijaste koji rade s Unity razvojnim okru�enjem.

Zaklju�no, glavni cilj ovog rada - spajanje robota na Unity i implementiranje multiplatformske aplikacije, dodatno podi�e mogu�nosti ROSa i Unityja koji zajedno mogu dovesti do vrlo zanimljivih i korisnih rezultata.