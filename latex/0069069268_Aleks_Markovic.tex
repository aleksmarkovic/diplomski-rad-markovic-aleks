% !TeX encoding = windows-1250

\input{tex_aux/rithesis_preamble}

% pomocu \includeonly moze se kompajlirati samo odredjeno poglavlje, da se skrati vrijeme kompajliranja, dok se ne isprave pogreske u tom poglavlju npr.:

%\includeonly{Poglavlje_1}

\usepackage{listings}
\lstset{frame=tb,
	language=Java,
	aboveskip=3mm,
	belowskip=3mm,
	showstringspaces=false,
	columns=flexible,
	basicstyle={\small\ttfamily},
	numbers=none,
	numberstyle=\tiny\color{gray},
	keywordstyle=\color{blue},
	commentstyle=\color{dkgreen},
	stringstyle=\color{yellow},
	breaklines=true,
	breakatwhitespace=true,
	tabsize=3
}

\begin{document}

\frontmatter   % - ne dirati

% upisati naziv studija
\degreesubject{Diplomski studij ra�unarstva} % upisati odgovarajuci naziv studija

% upisati vrstu rada
\documenttype{Diplomski rad}  % Zavrsni rad ili Diplomski rad

\title{Upravljanje robotom i mapiranje okoline u Unity 3D
	 \\ (Robot control and mapping with Unity 3D)}   % upisati specificni naslov rada

\date{\MONTH~\the\year.}   % ne dirati - mjesec i godina ?e se upisati sami

\author{Aleks Markovi�}  % upisati svoje ime i prezime
\jmbag{0069069268}  % upisati vlastiti JMBAG
\maketitle		% ne dirati

%\makecopyright

% Okruzenje za pisanje posvete. Maknuti komentare ukoliko se ?eli napisati posvetu.
%\begin{dedication}
%	Ovo je posveta nekome
%\end{dedication}

\mentor{prof.dr.sc.~Kristijan Lenac}   % zamijeniti podacima o svojem mentoru
\maketitleabstract

% kreira mjesto za umetnuti stranicu s opisom zadatka - ne dirati
\begin{assignmentpage}
\end{assignmentpage}

% kreira mjesto za umetnuti stranicu s izjavom o samostalnoj izradbi zadatka - ne dirati
\begin{honestystatementpage}
	% !TeX encoding = windows-1250

{ \large 
\vspace{15pt}
% prilagodite ovu izjavu s obzirom na potrebni rod imenice (izradio ili izradila)
%::::::::::::::::::::::::::::::::::::::::::::::::::::::::::::::::::::::::::::::::
Izjavljujem da sam samostalno izradio ovaj rad sukladno �lanku 9. pravilnika o diplomskom radu. \vspace{7cm} \\ 

%::::::::::::::::::::::::::::::::::::::::::::::::::::::::::::::::::::::::::::::::
% ovo ispod nema potrebe dirati!
%:::::::::::::::::::::::::::::::::::::::::::::::::::::::::::::::::::::::::::::::::::
\noindent Rijeka, \MONTH~\thisyear.   
\hspace{5.5cm}
	\verb|_______________|  % pomo?u duzine ove crte regulirati potrebnu sirinu crte za potpis

\begin{flushright}
	\vspace{-15pt}
	Ime Prezime 
	\verb|      |   % pomocu praznih mjesta unutar | | crta, regulirati poravnjanje duzine imena i prezimena s gornjom crtom za potpis
\end{flushright}
%:::::::::::::::::::::::::::::::::::::::::::::::::::::::::::::::::::::::::::::::::::

} % \large
\end{honestystatementpage}

% Okruzenje za pisanje zahvale
\begin{acknowledgments} % staviti znak komentara ukoliko se ne stavlja tekst zahvale
	\vspace{5pt}

\begin{flushleft}
\noindent Zahvaljujem xxxxxx na podr�ci tijekom pisanja ovoga rada i korisnim raspravama i savjetima. Zahvaljujem xxxxx na podr�ku tijekom studiranja.
\end{flushleft}  
\end{acknowledgments}

% kreiranje popisa sadrzaja, slika i tabela - ni?ta ne dirati
\tableofcontents
\listoffigures
\listoftables

\mainmatter		% ne dirati

% Ovdje pomo?u include funkcije ucitavati kreirana poglavlja. Poglavljima dajte logicna imena s obzirom na sadrzaj prikazan u njima (bez razmaka u imenu).
%\include{Intro}   % ovo poslije staviti pod komentar kada se nau?i koristiti
% !TeX encoding = windows-1250
\chapter{Uvod}
\qquad Robotika kao znanost moderne sada�njice, sveprisutna je u svakom segmentu na�eg �ivota, �to ukazuje potrebu i nu�nost izrade aplikacija za upravljanje i interakciju sa robotima. 

S obzirom da je cilj ovog diplomskog rada napraviti funkcionalnu aplikaciju za upravljanje robotima i mapiranje okoline, za izradu iste koristit �e se Unity razvojno okru�enje s kojim �e se jednostavnije ostvariti postavljeni cilj izrade univerzalnog i multiplatformskog softvera za upravljanje vi�e vrsta robota.  

Kao glavni alat za spajanje i upravljanje na robota koristi se ROS 1 (Robotski Operacijski  Sustav) (dalje u tekstu: ROS), a da bi se omogu�ila komunikacije  izme�u ROSa,  tj.   robota  i Unity-ja, koristiti �e se ROS\# knji�nicu. Implementacija i testiranje robota  provest �e se uz  popularni Turtlebot  3, za �to �e se koristiti simulirano okru�enje, odnosno simulacija Turtlebota i njegovog modela.

S obzirom da je ROS\# noviji alat, s prvom verzijom razvijenom po�etkom 2018. godine, nema previ�e podr�ke diljem interneta �to �ini ovaj rad zanimljivijim, smislenijim i izazovnijim.

Rad je razra�en u 4 glavna dijela, po�ev�i s opisom problema pa do analize kori�tenja hardverskih i programskih alata te do implementiranog rje�enja i njegovih rezultata.

\clearpage  % dati neko logicno ime umjesto ``Poglavlje_1''
\chapter{Softverski alati}
\qquad Prije samog rije�avanja problematike kako napraviti navedenu aplikaciju, potrebno je objasniti �to su i kako funkcioniraju kori�teni softverski alati. Definirati �e se i koji su preduvjeti, tj. knji�nice ili alati koji svaki od njih zahtjeva da se mo�e odraditi funkcija koja im se zada za prethodno navedenu svrhu.
\section{ROS}
\qquad Robotski Operacijski Sustav (ROS)
\section{Unity 3D}
\section{ROS\#}

% itd.

%%%%  POGLAVLJE ZAVRSENO  %%%%%
 
%\include{Poglavlje_3} % itd.


\include{Literatura}  % ovo je ime Bibtex datoteke koju korisnik kreira


%:::::::::: ukljucenje popisa kratica u tekst ::::::::
% Blok linija koda ispod ovoga generira ukljucenje popisa kratica u tekstu. Za uporabu, vidjeti Upute.
% Nije obavezno. Ako se ne zeli koristiti, onda ovaj blok staviti u komentar pomocu znaka %
\printglossary[type=\acronymtype]
\pagestyle{plain}
\begin{glossary}{Longest string}
	\input{Kratice}
\end{glossary}


%:::::::::::: blok za definiranje Sazetka/Abstracta rada 
\begin{abstract}
	% !TeX encoding = windows-1250
\vspace{5pt}

%:::::::::::::::::::::::::::::::::::::::::::::::::::::
%:::::::::::: HRVATSKI :::::::::::::::::::::::::::::::
\noindent
Cilj ovog rada bio je napraviti funkcionalnu aplikaciju za upravljanje robotom i mapiranje prostora koriste�i Unity razvojno okru�enje. Za spajanje ROS sustava i Unity razvojnog okru�enja kori�ten je ROS\# alat. Kartiranje se vr�i paralelno uz upravljanje robotom. Izra�uje se 2D karta koja se prikazuje ispod robota. Omogu�en je prikaz podataka s laserskog skena u obliku sfera i prikaz iz prvog lica kamere robota. U svrhu istra�ivanja napravljeno je da se u slu�aju dvaju robota u simulaciji mo�e mijenjati pogled kamere s jednog robota na drugi. Implementirano je i generiranje 3D karte, ali u zasebnoj sceni. Aplikacija je funkcionalna na tri platforme: Windows, Linux i Android. 
%:::::::::::::::::::::::::::::::::::::::::::::::::::::

\vspace{5pt}
%
\noindent \textbf{\textit{Klju�ne rije�i} --- ROS, Unity, ROS\#, upravljanje robotom, kartiranje} 

%:::::::::::: KRAJ HRVATSKOG DIJELA :::::::::::::::::::


%::::::::::::::::::::::::::::::::::::::::::::::::::::::
%:::::::::::: ENGLESKI ::::::::::::::::::::::::::::::::

%\vspace{-10pt}
\section*{Abstract}
\vspace{-10pt}
The objective of this thesis was to make a functional application for robot control and mapping using the Unity game engine. The connection of ROS and Unity game engine has been made with the ROS\# tool. Mapping is done in parallel with the robot control. A 2D map is being made which is shown under the robot. Laser scan data is displayed in a form of spheres and a first person view from the robot camera was also enabled. For research purposes in the case of two robots in the same simulation, a feature was created in order to switch the camera view from one robot to another. 3D mapping is also implemented, but on a separate scene. The application is functional on three platforms: Windows, Linux and Android. 
%:::::::::::::::::::::::::::::::::::::::::::::::::::::::

\vspace{5pt}
%
\noindent \textbf{\textit{Keywords} --- ROS, Unity, ROS\#, robot control, mapping}

%::::::::::::::::::::::::::::::::::::::::::::::::::::::
%:::::::::::: KRAJ ENGLESKOG DIJELA :::::::::::::::::::

	  % sazetak rada i kljucne rijeci na HR i EN
\end{abstract}


%:::::::::::: PRILOZI (neobavezno) ::::::::::::::::::::
% ispod \appendix zaglavlja pomocu \include dodati poglavlja s prilozima
% ukoliko nemate priloga, ovaj blok linija staviti u komentar
\appendix
\include{Prilog_1}  % dati neko suvislo ime umjesto ovoga
%\include{Prilog_2}  % itd.


\end{document}
