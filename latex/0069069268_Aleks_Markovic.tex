% !TeX encoding = windows-1250

\input{tex_aux/rithesis_preamble}

\usepackage{listings}
\usepackage{pdfpages}
%\usepackage{pstricks}

\lstset{frame=tb,
	language=Java,
	aboveskip=3mm,
	belowskip=3mm,
	showstringspaces=false,
	columns=flexible,
	basicstyle={\small\ttfamily},
	numbers=none,
	numberstyle=\tiny\color{gray},
	keywordstyle=\color{blue},
	commentstyle=\color{dkgreen},
	stringstyle=\color{yellow},
	breaklines=true,
	breakatwhitespace=true,
	tabsize=3
}

\begin{document}

\frontmatter   % - ne dirati

% upisati naziv studija
\degreesubject{Diplomski studij ra�unarstva} % upisati odgovarajuci naziv studija

% upisati vrstu rada
\documenttype{Diplomski rad}  % Zavrsni rad ili Diplomski rad

\title{Upravljanje robotom i mapiranje okoline u Unity 3D}   % upisati specificni naslov rada

\date{\MONTH~\the\year.}   % ne dirati - mjesec i godina ?e se upisati sami

\author{Aleks Markovi�}  % upisati svoje ime i prezime
\jmbag{ 0069069268}  % upisati vlastiti JMBAG
\maketitle		% ne dirati

%\makecopyright

% Okruzenje za pisanje posvete. Maknuti komentare ukoliko se ?eli napisati posvetu.
%\begin{dedication}
%	Ovo je posveta nekome
%\end{dedication}

\mentor{izv. prof. dr. sc.~Kristijan Lenac}   % zamijeniti podacima o svojem mentoru
\maketitleabstract

% kreira mjesto za umetnuti stranicu s opisom zadatka - ne dirati
%\begin{assignmentpage}
%\end{assignmentpage}

\includepdf[noautoscale]{zadatak.pdf}

% kreira mjesto za umetnuti stranicu s izjavom o samostalnoj izradbi zadatka - ne dirati
\begin{honestystatementpage}
	% !TeX encoding = windows-1250

{ \large 
\vspace{15pt}
% prilagodite ovu izjavu s obzirom na potrebni rod imenice (izradio ili izradila)
%::::::::::::::::::::::::::::::::::::::::::::::::::::::::::::::::::::::::::::::::
Izjavljujem da sam samostalno izradio ovaj rad sukladno �lanku 9. pravilnika o diplomskom radu. \vspace{7cm} \\ 

%::::::::::::::::::::::::::::::::::::::::::::::::::::::::::::::::::::::::::::::::
% ovo ispod nema potrebe dirati!
%:::::::::::::::::::::::::::::::::::::::::::::::::::::::::::::::::::::::::::::::::::
\noindent Rijeka, \MONTH~\thisyear.   
\hspace{5.5cm}
	\verb|_______________|  % pomo?u duzine ove crte regulirati potrebnu sirinu crte za potpis

\begin{flushright}
	\vspace{-15pt}
	Ime Prezime 
	\verb|      |   % pomocu praznih mjesta unutar | | crta, regulirati poravnjanje duzine imena i prezimena s gornjom crtom za potpis
\end{flushright}
%:::::::::::::::::::::::::::::::::::::::::::::::::::::::::::::::::::::::::::::::::::

} % \large
\end{honestystatementpage}

% Okruzenje za pisanje zahvale
\begin{acknowledgments} % staviti znak komentara ukoliko se ne stavlja tekst zahvale
	\vspace{5pt}

\begin{flushleft}
\noindent Zahvaljujem xxxxxx na podr�ci tijekom pisanja ovoga rada i korisnim raspravama i savjetima. Zahvaljujem xxxxx na podr�ku tijekom studiranja.
\end{flushleft}  
\end{acknowledgments}

% kreiranje popisa sadrzaja, slika i tabela - ni?ta ne dirati
\tableofcontents
\listoffigures
\listoftables

\mainmatter		% ne dirati

% Ovdje pomo?u include funkcije ucitavati kreirana poglavlja. Poglavljima dajte logicna imena s obzirom na sadrzaj prikazan u njima (bez razmaka u imenu).
%\include{Intro}   % ovo poslije staviti pod komentar kada se nau?i koristiti
% !TeX encoding = windows-1250
\chapter{Uvod}
\qquad Robotika kao znanost moderne sada�njice, sveprisutna je u svakom segmentu na�eg �ivota, �to ukazuje potrebu i nu�nost izrade aplikacija za upravljanje i interakciju sa robotima. 

S obzirom da je cilj ovog diplomskog rada napraviti funkcionalnu aplikaciju za upravljanje robotima i mapiranje okoline, za izradu iste koristit �e se Unity razvojno okru�enje s kojim �e se jednostavnije ostvariti postavljeni cilj izrade univerzalnog i multiplatformskog softvera za upravljanje vi�e vrsta robota.  

Kao glavni alat za spajanje i upravljanje na robota koristi se ROS 1 (Robotski Operacijski  Sustav) (dalje u tekstu: ROS), a da bi se omogu�ila komunikacije  izme�u ROSa,  tj.   robota  i Unity-ja, koristiti �e se ROS\# knji�nicu. Implementacija i testiranje robota  provest �e se uz  popularni Turtlebot  3, za �to �e se koristiti simulirano okru�enje, odnosno simulacija Turtlebota i njegovog modela.

S obzirom da je ROS\# noviji alat, s prvom verzijom razvijenom po�etkom 2018. godine, nema previ�e podr�ke diljem interneta �to �ini ovaj rad zanimljivijim, smislenijim i izazovnijim.

Rad je razra�en u 4 glavna dijela, po�ev�i s opisom problema pa do analize kori�tenja hardverskih i programskih alata te do implementiranog rje�enja i njegovih rezultata.

\clearpage  
% !TeX encoding = windows-1250
\chapter{Opis problema}

\qquad Cilj ovog rada je napraviti funkcionalnu aplikaciju za upravljanje robotom i mapiranje okoline koriste�i Unity razvojni program. Jedna od bitnih stavki u navedenom zadatku jest komunikacija i razmjena podataka izme�u ROSa i Unityja. U tu svrhu koristit �e se ROS\# knji�nica koja sadr�i set alata za adaptaciju i interakciju ROSa s Unityjem.

\section{ROS\#}

\qquad ROS\# je odabran iz razloga �to on ve�inski rje�ava problem komunikacije izme�u ROSa i Unitya, no ne sasvim. Za kompleksnije stvari kao �to je mapiranje, potrebno je pro�iriti mogu�nosti ROS\#a na na�in da se omogu�i intuitivno i efikasno �itanje ROS poruka. Tako�er treba rje�iti problem kako iskoristiti i prikazati dobivene podatke u Unity ekosustavu. 

Nakon instalacije potrebnih alata, potrebno je iste konfigurirati na na�in da se mogu koristiti sa ROS\#om i Unityjem. U to je uklju�ena i potreba da se ROS\# knji�nica uvede u Unity. Zavr�etkom ovih osnovnih koraka omogu�eno je kretanje s implementacijom glavnog cilja - razvoj aplikacije za upravljanje robotom i mapiranje okoline. 

\subsection{Alternativni alati}

\qquad Uz ROS\# postoje jo� dva potencijalna alata koji slu�e za istu svrhu:
\begin{enumerate}
	\item \textbf{Unity Robotics Hub} - Slu�beni Unity alat koji se razvija od strane tvrtke \emph{Unity Technologies}. Navedeni alat je manje popularan od ROS\# alata no ubrzano raste. Dijelovi Unity Robotics Huba su proiza�li iz ROS\#a, a to su alat za uvoz URDF modela i TCP konektor koji su dodatno modificirani od strane Unityja. Iako navedeni alat jo� nije dostigao ROS\# po mogu�nostima i popularnosti, aktivno se razvija i izra�uju nove ina�ice pa bi u skoroj budu�nosti mogao postati vode�i alat za spajanje ROSa i Unityja. \cite{unityroboticshub}
	\item \textbf{Unity ROS} - Alat razvijen od strane privatne osobe koji je zadnje a�uriran prije tri godine te nije do�ivio veliku popularnost i kori�tenje od strane ROS i Unity zajednice. Razlog toga jest lo�a podr�ka i limitirane funkcionalnosti alata. \cite{unityros}
\end{enumerate}

Iako svi navedeni alati imaju sli�ni temelj za uspostavljanje komunikacije izme�u ROSa i Unityja, ROS\# i dalje vodi na ljestvici popularnosti svojom fleksibilno��u, ve�im mogu�nostima te podr�kom zajednice i programera.


\section{Unity i aplikacija}
\qquad Jednom kad Unity sadr�i funkcionalnu ROS\# knji�nicu, dobiva se pristup alatu za uvo�enje URDF modela robota, �to je ujedno i prvi korak u izradu konkretne aplikacije. Potrebno je uvesti model kori�tenog robota �to uklju�uje i njegove fizi�ke i motori�ke karakteristike kako bi se ga moglo ispravno prikazati u Unityju te upravljati istim. U tu svrhu potrebno je pokrenuti \emph{publish\_description\_turtlebot.launch} datoteku koja je detaljnije opisana u Dodatku A ovog rada. Prije samog pokretanja ove \emph{launch} datoteke potrebno je navesti u kojem �e se ROS paketu i gdje unutar paketa prona�i i u�itati URDF model �eljenog robota. Pokretanjem \emph{launch} datoteke u�itava se URDF model robota te se kroz \emph{file\_server} alat objavljuje kao ROS �vor kojega ROS\# URDF alat mo�e pro�itati i uvesti u Unity. Ovo je trenutno jedini na�in za efikasno uvo�enje modela robota u Unityju.

S obzirom da je Unity poznat kao dobar alat za razvijanje softvera za vi�e platforma odjednom, isti �e se iskoristiti na na�in da se ciljna aplikacija napravi kako bi ista bila kompatibilna za:
\begin{enumerate}
	\item Linux Ubuntu - Na prvom mjestu se nalazi Ubuntu iz razloga �to se cijeli rad radi na ovom operacijskom sustavu, prvobitno zbog ROSa.
	\item Android - Jedan od ve�ih zahtjeva danas je imati mobilnu aplikaciju. Posebno je to slu�aj za upravljanje nekim robotom iz razloga �to je no�enje prijenosnog ra�unala te�e. Mobilna aplikacija za upravljanje robotom olak�ava i pobolj�ava korisni�ko iskustvo pri upravljanju robota.
	\item Windows - Kao najkori�teniji operacijski sustav danas, prikladno je imati aplikaciju i za njega.
\end{enumerate}
Mac OS aplikaciju je tako�er mogu�e napraviti, no zbog tehni�ke ograni�enosti presko�it �e se.

Pri izradi aplikacije za pojedinu platformu, potrebne su neke dorade da bi aplikacija radila na toj platformi. Zahvaljuju�i Unityju i Mono frameworku koji Unity koristi (.NET Standard 2.0) navedene dorade su dovoljno sitne da je prelazak na drugu platformu relativno bezbolan.

Unato� tome, neke zna�ajke znaju bit nekompatibilne. Kao op�i primjer mogu se navesti na�ini interakcije korisnika s aplikacijom, npr. mobilni �e ure�aj imati na raspolaganju razne senzore, zaslon na dodir i sl., ali ne�e imati tipkovnicu i mi� (osim kod eksplicitnog spajanja dodatnog hardvera), dok se osobno ra�unalo uvijek koristi tipkovnicom i mi�em ali ve�inom nema iste senzore kao mobilni ure�aj, a i zaslon na dodir nije uvijek prisutan. Iz tog se razloga ne mo�e uvijek implementirati isti na�in interakcije korisnika s aplikacijom za razne platforme i ure�aje a da bude uvijek funkcionalan.

Jo� jedan mogu�i problem kod implementacije svih potrebnih zna�ajki jest mogu�nost prevelikog protoka podataka, gdje bi moglo do�i do efekta uskog grla (eng. bottleneck). Potrebno je dobro se informirati o tome koliko �e potrebna komunikacija izme�u ROSa i Unityja za pojedinu zna�ajku slati podataka (koliko svaka ROS tema �alje podataka i po kojoj frekvenciji). Ovisno o tome, postoji mogu�nost da treba:  promjeniti kvalitetu emitirane slike s kamere, smanjiti frekvenciju slanja podataka o mapiranju ili �ak i promjeniti na�in stvaranja podataka (npr. druga metoda mapiranja). Postoje slu�ajevi koje se ne�e mo�i procjeniti dok se ne iste poku�a implementirati.

\subsection{Pogled s kamere}

\qquad Implementacija prikaza pogleda s kamere robota u realnom vremenu treba biti napravljena na na�in da se slika nesmetano prikazuje na ekranu, bez obzira na razlu�ivost. Postoje vi�e na�ina na koje se mo�e taj problem rje�iti, od kojih su dva prikladna:

Postaviti 3D objekt ispred robota koji slu�i kao platno gdje se postavlja slika s kamere. Platno bi trebalo biti fiksne pozicije u odnosu na poziciju i orijentaciju robota.

Drugi na�in je napraviti 2D Canvas na kojega bi se moglo postaviti komponentu slike s kamere robota. Oba su na�ina izvediva, no potrebno ih je isprobati i uviditi mogu�e pote�ko�e koje bi mogle biti smetnja pri skaliranju slike na razli�ite rezolucije i ure�aje.

\subsection{Upravljanje robotom}

\qquad Upravljanje robotom podrazumijeva upravljanje njegovim motori�kim sposobnostima. U ovom �e se radu to ograni�iti na kretanje robota kroz prostor. Potrebno je implementirati da se upravljanje robotom mo�e izvr�avati na svim navedenim platformama. Ovdje najve�i problem predstavlja prija�nje navedene mogu�e razlike u hardveru ure�aja (ra�unalo i mobilni ure�aj). Iz tog razloga postoji mogu�nost da �e se trebati implementirati vi�e metoda upravljanja robotom.

\subsection{Mapiranje}

\qquad Najve�i implementacijski problem predstavlja mapiranje. Po�to je kori�teni robot edukacijske svrhe, isti nema kompleksne komponente za interakciju s prostorom ali opremljen je laserskim skenom i kamerom. Sukladno tome, potrebno je istra�iti dostupne metode i omogu�iti 2D i 3D mapiranje. Postoje popularni alati u ROS sustavu koji to omogu�uju uz pretpostavku da je robot opremljen potrebnim komponentama, npr. laserski skener za 2D mapiranje ili kamera s mogu�nosti 3D percepcije (percepcija dubine prostora) za 3D mapiranje prostora. 

Kada se mapiranje uspje�no pokrene u ROS sustavu, potrebno je rezultiraju�e podatke (mapu) mo�i prenijeti u Unity. Ovdje �e trebati analizirati rezultiraju�e podatke da bi se do�lo do zaklju�ka gdje i kako se trebaju procesirati. Podaci �e se svakako morati procesirati u Unityju prije kori�tenja, no mogu�e je da �e se trebati prije i formatirati na odre�eni na�in da bi dohvat podataka u Unityju bio efikasniji. Za formatiranje podataka najprikladnije je napraviti ROS �vor kojega se mo�e isprogramirati da �ita i formatira podatke ROS tema koje objaljvuje alat za mapiranje kojeg se bude odabralo kao najprikladnijim. Formatirane podatke se tada mo�e ponovno objaviti na isti na�in.

Kori�tenje dobivenih podataka u Unityju podrazumijeva kako �e se te podatke pretvoriti u vizualni element u aplikaciji. U slu�aju 3D mapiranja cijeli �e se proces trebati optimizirati iz razloga �to su 3D elementi ra�unski skuplji - svaki 3D element u aplikaciji je zasebni objekt kojeg treba stvoriti i dr�ati u memoriji. Dok 2D elementi se mogu postavljati kao npr. slike ili teksture �to drasti�no smanjuje potrebu za resursima ra�unala jer predstavljaju jedan objekt ili komponentu objekta u aplikaciji.
\clearpage 
% !TeX encoding = windows-1250
\chapter{Specifikacije robota}

\qquad U ovom se radu koristio trenutno najpopularniji svijetski robot za edukacijske svrhe - Turtlebot 3. Od svoje dvije glavne ina�ice, Burger i Waffle, kori�ten je Waffle (slika \ref{fig:turtlebot}) zbog boljih mogu�nosti od kojih je najbitnija bila kamera. U prvoj fazi rada se koristio Waffle Pi, ina�ica koja koristi Raspberry Pi kao ugra�eno ra�unalo, ali se u drugoj fazi rada odlu�ilo koristiti obi�nu verziju Waffle-a. Pi verzija koristi Raspberry Pi kameru koja nema 3D mogu�nosti dok obi�na verzija koristi Intel� Joule� 570x s Intel� Realsense� R200 kamerom koja omogu�ava 3D percepciju prostora. Obi�na Waffle ina�ica se vi�e ne proizvodi iz razloga �to je Intel prestao s proizvodnjom Intel Joule-a, no ostali Turtlebot-ovi su jo� uvijek a�urni.

\begin{figure}[!htbp]
	\begin{center}
		\includegraphics[height=6cm,width=8cm,keepaspectratio=true]{turtlebot}
		\caption{Turtlebot 3 - Waffle \cite{turtlebot}}
		\label{fig:turtlebot}
	\end{center}
\end{figure}

Bitne karakteristike kori�tenog modela Turtlebot-a su sljede�e: 
\begin{enumerate}
	\item Intel� Joule� 570x ugra�eno ra�unalo, koje slu�i za pokretanje ROS sustava i svih ROS zna�ajki (slika \ref{fig:inteljoule}). Ovo se ra�unalo vrti na Linux operacijskom sustavu, koji koristi 1.7GHz Atom T5700 procesor, 16 GB ugra�ene memorije, 4 GB LPDDR4 RAM memorije, grafi�kim procesorom Intel� HD Graphics: HDMI 1.4B i ugra�enim 802.11ac Wi-Fi modulom. \cite{inteljoule}
	
	\begin{figure}[!htbp]
		\begin{center}
			\includegraphics[height=6cm,width=8cm,keepaspectratio=true]{intelJoule}
			\caption{Intel� Joule� 570x na svojoj popratnoj plo�i \cite{inteljouleimg}}
			\label{fig:inteljoule}
		\end{center}
	\end{figure}

	\item Laserski skener LDS-01 (slika \ref{fig:lds01}) skenira prostor u 360 stupnjeva i mogu�uje SLAM mapiranje i navigaciju. Maksimalni domet od 3,5 metra i brzina skeniranja od 300 okretaja po minuti �ini ga prikladnim za Turtlebot-a. \cite{lds01}
	
	\clearpage
	
	\begin{figure}[!htbp]
		\begin{center}
			\includegraphics[height=6cm,width=8cm,keepaspectratio=true]{lds01}
			\caption{LDS-01 laserski skener \cite{lds01}}
			\label{fig:lds01}
		\end{center}
	\end{figure}

	\item Intel� Realsense� R200 kamera (slika \ref{fig:intelcamera}) sa zna�ajkom 3D percepcije omogu�uje kori�tenje alata za izgradnju 3D mape. Njezine male dimenzije od samo 101.6mm x 9.6mm x 3.8mm omogu�uju montiranje na Turtlebot te mala rezolucija od 480 x 360 je prikladna za teku�i prijenos slike zbog manjeg protoka podataka. \cite{intelcamera}
			
	\begin{figure}[!htbp]
		\begin{center}
			\includegraphics[height=6cm,width=8cm,keepaspectratio=true]{intelRealSense}
			\caption{Intel� Realsense� R200 kamera \cite{intelcameraimage}}
			\label{fig:intelcamera}
		\end{center}
	\end{figure}

	\clearpage
	
	\item Te�inom od 1.8kg i veli�inom 281mm x 306mm x 141mm dosti�e brzinu od 0.26m/s s brzinom okretanja od 104.27 stupnjeva po sekundi.
	\item S maksimalnom visinom penjanja od manje od deset centimetara ograni�en je na ravne, ve�inom zatvorene prostore.
\end{enumerate} 
\chapter{Softverski alati}
\qquad Prije samog rije�avanja problematike kako napraviti navedenu aplikaciju, potrebno je objasniti �to su i kako funkcioniraju kori�teni softverski alati. Definirati �e se i koji su preduvjeti, tj. knji�nice ili alati koji svaki od njih zahtjeva da se mo�e odraditi funkcija koja im se zada za prethodno navedenu svrhu.
\section{ROS}
\qquad Robotski Operacijski Sustav (ROS)
\section{Unity 3D}
\section{ROS\#}

% itd.

%%%%  POGLAVLJE ZAVRSENO  %%%%%
 
% !TeX encoding = windows-1250
\chapter{Opis rje�enja}

\qquad Nastavno na dosad navedeno opisat �e se aplikacija i njezina struktura, kako je ista napravljena i na kojem principu funkcionira. Tri glavne komponente ovog sustava, kao �to je ve� navedeno, su: ROS, ROS\# i Unity. Na slici \ref{fig:maindiagram} prikazan je dijagram kako sustav izgleda gdje su nam navedene komponente glavni dijelovi dijagrama.

\qquad Prvi korak sustava jest pokretanje Gazebo simulacije u ROS-u gdje se simulira robota i njegovo okru�enje. Alternativa u pravom svijetu bi bila ista osim �to ne bismo trebali pokretati Gazebo ve� pravog robota na kojemu se tako�er izvr�ava ROS.

\qquad Drugi korak jest Unity odnosno Editor i njegove standalone te mobilne aplikacije koje su implementirane i adaptirane sa istom svrhom i mogu�nostima. �to zna�i da se mo�e bilo koja aplikacija pokrenuti i spojiti na navedeni robotski sustav.

\qquad ROS\# paket, odnosno most koji spaja ROS i Unity pomo�u RosBridgeClient-a otvara vezu izme�u prethodno navedena dva koraka, a koji funkcionira na temelju pretplate i izdavanja poruka na ROS teme (eng. topic). Jednom kad se veza otvorila, Unity se mo�e pretpla�ivati na ROS teme i izdavati nove poruke na ROS teme.

\qquad Na slici \ref{fig:maindiagram} prikazan je glavni princip prethodno navedenih mogu�nosti:
\begin{enumerate}
	\item Unity se pretpla�uje na ROS teme - povratne informacije, npr. odometrija, slika kamere, laserscan,
	\item Unity izdaje nove poruke na ROS teme - kontrolne poruke, npr. upravljanje robotom.	
\end{enumerate}

\begin{figure}[!htbp]
	\begin{center}
		\includegraphics[height=12cm,width=21cm,keepaspectratio=true]{diagram}
		\caption{Dijagram sustava \cite{rossharpgazebosim}}
		\label{fig:maindiagram}
	\end{center}
\end{figure}
\clearpage

\section{Opis su�elja i Unity objekata}

\qquad U ovoj �e se sekciji opisati koji su i �emu slu�e glavni objekti u kori�tenim Unity scenama.

\subsection{Meni}
\qquad Napravljen je jednostavni inicijalni meni s nekoliko polja za upis:
\begin{enumerate}
	\item IP adresa - polje za unos IP adrese ROS-a, tj. robota. IP adresu, ako je nepromjenjiva, je dovoljno upisati jednom jer se sprema pomo�u Unity zna�ajke \emph{PlayerPrefs} koja slu�i za spremanje igra�evih (korisni�kih) postavka.
	\item Prefiks robota - opcionalno polje u slu�aju da postoji vi�e robota u simulaciji. 
\end{enumerate}

Postoje i dva gumba gdje svaki vodi na jednu scenu gdje je implementirana neka zna�ajka. Na svakoj od ovih scena se vr�i spajanje na ROS i zapo�inje interakcija:
\begin{enumerate}
	\item Gumb koji vodi na scenu (dalje u tekstu scena 1) gdje je implementirana kontrola robota, pogled u prvom licu s kamere, 2D mapiranje i prikaz podataka laserskog skena.
	\item Gumb koji vodi na scenu (dalje u tekstu scena 2) gdje se vr�i 3D mapiranje prostora.
\end{enumerate}

Sljede�e �e se navesti svi elementi (objekti) u sceni 1 te �emu slu�e.

\subsection{Ros Connector}

\qquad Na sceni 1 najbitniji element, tj. objekt, jest \textbf{RosConnector} koji prvobitno slu�i za otvaranje veze prema ROS-u, ali i za sadr�avanje svih skripta kojima je zada�a pretplata ili izdavanje poruka na ROS temu. \textbf{RosConnector} sadr�i sljede�e relevantne skripte:
\begin{enumerate}
	\item \emph{Image Subscriber} koji se pretpla�uje na jednu ROS temu kamere, u ovom slu�aju se koristi \emph{/camera/rgb/image\_raw/compressed}.
	\item \emph{Laser Scan Subscriber} koji se pretpla�uje na ROS temu laserskog skena (\emph{/scan}). 
	\item \emph{Map Subscriber} koji se pretpla�uje na ROS temu \emph{/map} koju izdaje �vor \emph{slam\_gmapping}. 
	\item \emph{Odometry Subscriber} koji se pretpla�uje na temu \emph{/odom} i dohva�a odometrijske podatke robota.
	\item \emph{Joystick Publisher} izdaje poruke na \emph{/joy} ROS temu, koja slu�i kao joystick kontrola robota.
	\item \emph{Twist Publisher Static} u�itava ulaz gumbova na glavnoj sceni za upravljanje smjerom robota. Isti izdaje poruke na \emph{/cmd\_vel} ROS temi koja upravlja linearnom i kutnom brzinom robota.
\end{enumerate}

\subsection{Plane}

\qquad Plane je 3D objekt koji slu�i kao podloga (pod ili teren) robotu. Bez podloge bi se trebala isklju�iti svojstvo gravitacije u Unity-ju, no na taj na�in se mo�e podloga iskoristiti za projekciju generirane mape okoline robota. Na na�in kada stavimo prikaz scene iznad navedenog objekta i robota, dobiva se efekt robota koji ide kroz mapu koja se paralelno izra�uje.

\subsection{Model robota}

\qquad Model robota se prikazuje kao zaseban objekt u sceni, u ovom slu�aju objekt \textbf{turtlebot3\_waffle\_pi}. Isti je generiran prija�nje navedenim alatom za uvoz URDF modela i sadr�i podobjekte po svojstvima robota kao na slici \ref{fig:robotmodel}.

\begin{figure}[!htbp]
	\begin{center}
		\includegraphics[height=8cm,width=14cm,keepaspectratio=true]{robotModel}
		\caption{Model robota u Unity-ju}
		\label{fig:robotmodel}
	\end{center}
\end{figure}

Pomi�ni podobjekti, npr. kota�i, se vrte isto kao i u simulaciji zahvaljuju�i pomo�nim skriptama ROS\#-a koje u�itavaju stanja kota�a sa simulacije. 

\subsection{Canvas}

\qquad \textbf{Canvas} u Unity-ju je 2D objekt koji ve�inom slu�i za slaganje grafi�kih komponenta su�elja, npr. gumbovi, tekst ulazi i sl. Jedna od najve�ih prednosti Canvas-a je responzivnost. Canvas-u se mo�e definirati bazna rezolucija, koja se onda mo�e skalirati ovisno o rezoluciji ure�aja gdje je aplikacija pokrenuta. Isto vrijedi za sve elemente u Canvas-u. Elementi se mogu "usidriti" u nekom kutu ekrana, mogu se �iriti i smanjivati ovisno o rezoluciji, mijenjati poziciju ovisno o veli�ini ekrana i sl.

\qquad U ovoj aplikaciji Canvas se sastoji od sljede�ih elemenata:
\begin{enumerate}
	\item \emph{Camera Image} kao �to ime govori, ovo je element Canvas-a gdje nam se prikazuje slika s kamere. Slika kamere se u�itava kao tekstura te ista preslikava kao \emph{Sprite} na navedeni element koji sadr�i polje za sliku. Element je pro�iren po cijeloj du�ini i �irini Canvas-a, tako da se ovisno o rezoluciji ekrana skalira.
	\item \emph{Control} je objekt koji sadr�i podobjekte koji su gumbovi za odre�ivanje linearne i kutne brzine robota. 
	\item \emph{Laserscan Button} je gumb koji uklju�uje i isklju�uje prikaz objekata generiranih laserskim skenom.
	\item \emph{Camera Button} je gumb koji mjenja prikaz kamere (topografski prikaz mape i robota te prikaz u prvom licu s kamere robota).
	\item \emph{2D Map Button} je gumb koji uklju�uje i isklju�uje crtanje i prikaz 2D mape na \emph{Plane} objekt.
	\item \emph{Switch Robot Camera} mjenja prikaz kamere u prvom licu na drugog robota (u slu�aju da ima vi�e robota u simulaciji). Ako u simulaciji postoji samo jedan robot, tada je taj gumb isklju�en.
	\item \emph{FPS} koji prikazuje trenutni FPS (broj a�uriranja slike u sekundi) u sceni.
\end{enumerate}

\subsection{Multi Robot Control}

\qquad \emph{Multi Robot Control} objekt slu�i samo za pokretanje skripte koja omogu�uje prikaz kamera dva robota odjednom.
\clearpage

\subsection{Scena 2}

\qquad Scena 2 sadr�i iste elemente kao scena 1 osim �to od Canvas gumbova ima samo \emph{Control}, \emph{FPS} i novi \emph{3D Map Button} koji slu�i za aktiviranje i deaktiviranje crtanja 3D objekata u sceni - 3D mape.

\section{Zna�ajke i mogu�nosti rje�enja}

\qquad U sljede�oj �e se sekciji navesti i opisati zna�ajke i mogu�nosti koje se implementiralo u ovom radu.

\subsection{Kamere}

\qquad U sceni postoje dvije kamere. Kamera u Unity-ju ozna�ava kako se scena prikazuje. Ona mo�e biti iz razli�itih kuteva, sa razli�itom �irinom i dubinom snimanja, bilo kojom pozicijom snimanja i jo� mnogo postavka. Jedna �e kamera biti postavljena iznad 3D objekata, a druga �e biti namje�tena da snima Canvas - u ovom slu�aju pozicija kamere nije bitna jer fiksno snima 2D Canvas.

Prikaz kamere iz prvog lica - Prva zna�ajka na koju se nailazi prilikom pokretanja aplikacije je prikaz kamere iz prvog lica. Slika se u stvarnom vremenu dohva�a sa robota te se kao tekstura postavlja na prija�nje navedenom Canvas objektu.

Prikaz kamere iz pti�je perspektive - U ovoj se perspektivi prikazuje 3D model robota na 2D mapi koja se izra�uje s dobivenim podacima iz \emph{/map} ROS teme, gdje se izra�uje nova tekstura koja se precrtava na \emph{Plane} objekt. Tako�er su u ovoj perspektive vidljive sfere (tako�er 3D objekti) koji se crtaju shodno podacima iz \emph{/scan} ROS teme.


\subsection{Laserski sken}

\qquad ROS teme laserskog skena �alju poruke formata \emph{sensor\_msgs/LaserScan} u kojem se nalazi vi�e vrsta podataka, ali je u ovom slu�aju bitno float32 polje \emph{ranges}. Problem je �to ovo polje zna sadr�avati beskona�ne brojeve te tada ROS\# izbacuje gre�ku i ne nastavlja s procesiranjem tog polja. Iz toga razloga je bilo potrebno napraviti ROS �vor koji �e �itati temu laserskog skena, ROS tema \emph{/scan}, filtrirati podatke koji su beskona�ni te izdati novu ROS temu sa ispravljenim podacima. Dobiveni podaci se pomo�u ROS\# skripte \emph{Laser Scan Writer} i \emph{Laser Scan Visualizer} crtaju kao sfere u 3D prostoru oko robota. 

\subsection{2D mapiranje}

\qquad 2D mapiranje u Unity-ju se vr�i na na�in da se u�itavaju podaci iz ROS tema \emph{/map}. Ista �alje poruke tipa \emph{nav\_msgs/OccupancyGrid}. Najrelevantniji podaci ove teme su nam �irina i du�ina mape te int8 polje podataka u kojem se nalazi podaci vjerojatnosti u rasponu od 0 do 100 - vjerojatnost da je jedno skenirano polje zauzeto nekim objektom u prostoru. Kada se vjerojatnost ne mo�e izra�unati, u polju podataka bude broj -1.
Proces crtanja mape u Unity-ju je sljede�i:
\begin{enumerate}
	\item U�itava se du�ina i �irina mape.
	\item Inicijalizira se nova varijable za mapu koja �e biti tipa \emph{Color} i biti �e iste duljine kao i polje podataka iz \emph{/map} teme.
	\item Iterira se po tom dobivenom polju podataka te se ovisno o dobivenom podatku u novoinicijalizirano polje boja dodaje nova definirana boja.
	\item Nakon iteriranja je potrebno postaviti zastavicu (flag) u \emph{true} koja predstavlja da je potrebno nacrtati novu mapu na su�elju.
	\item Unity funkcija \emph{Update} koja se izvr�ava svako osvje�avanje ekrana, provjerava navedenu zastavicu i izra�uje novu teksturu na temelju definiranog polja boja, du�ine i �irine mape te se ista nova tekstura sprema na glavnu teksturu \emph{Renderer} komponente na \emph{Plane} objektu. 
\end{enumerate}

Kori�tena metoda mapiranja je \emph{SLAM (Simultana Lokalizacija i Mapiranje) gmapping} koja je bazirana na podacima laserskog skena i pozicije robota. Ista je kreator prija�nje navedenih \emph{Occupancy Grid} podataka (mre�a popunjenosti) tj. mape. Mapa se a�urira u intervalu od 4 sekunde.

Ispunjavanje novog polja boja u petlji je vrlo ne optimiziran pristup. U aplikaciji se osje�ao trzaj u izvr�avanju tijekom izrade mape, pa se kao rje�enje na to izvr�avanje funkcije koja sadr�i petlju izvr�ava kao novi \emph{Task} koji u C\# ozna�uje asinkronu operaciju na novoj dretvi. Time se drasti�no smanjio trzaj u aplikaciji.

\subsection{Upravljanje}

\qquad Upravljanje robotom se mo�e vr�iti na dva na�ina: 
\begin{itemize}
	\item Kori�tenjem gumbova na su�elju - Svaki gumb inkrementira ili dekrementira brzinu za definirani korak. Trenutno definirani korak za linearnu brzinu iznosti 0.05, a za kutnu 0.02. Svakom promjenom se na \emph{/cmd\_vel} ROS temi �alje poruka tipa \emph{geometry\_msgs/Twist} s novom brzinom. Poruka sadr�i Vector3 tip varijable za linearnu i kutnu brzinu. \emph{/cmd\_vel} ROS tema proslije�uje naredbe brzine samom robotu.
	\item Kombinacijom tipka (w, a, s, d) ili tipka strelica - Unity pomo�u ROS\# pomo�nih skripti koje u�itavaju promjenu u ulazu \emph{Input Manager-a}, pretvara ulaze u upravlja�ku naredbu za \emph{/joy} ROS temu. Nakon toga ROS\# �vor \emph{joy\_to\_twist} pretvara poruke \emph{/joy} ROS teme u poruke formata \emph{geometry\_msgs/Twist} te i �alje na ROS temu \emph{/cmd\_vel}. Unity \emph{Input Manager} je dio postavka gdje se mogu konfigurirati na�ini ulaza naredba u aplikaciju.
\end{itemize}

Samo se jedna od navedenih metoda mo�e koristiti odjednom iz razloga �to obe metode �alju poruke na istu ROS temu (\emph{/cmd\_vel}) pa �e u slu�aju kori�tenja obe metode, metoda joysticka ili tipka premostiti metodu gumbova. 

Povratna informacija s \emph{/odom} ROS teme se koristi za a�uriranje odometrijskih podataka robotskog modela u Unity prostoru.

\subsection{Prikaz s vi�e robota}

\qquad U svrhu istra�ivanja, omogu�eno je da se mogu prikazivati slike kamera u simulaciji gdje postoje dva robota iste vrste. Kao primjer je uzeta i adaptirana Turtlebot3 simulacija s vi�e robota.

U po�etnom izborniku potrebno je upisati prefiks ROS tema ispred teme robota, npr. postoji simulacija s vi�e Turtlebot-ova gdje svaki Turtlebot ima svoje ROS teme koje su ina�e istog naziva, njima je potrebno dodati prefiks ispred naziva ROS teme da bi se znalo na kojeg Turtlebot-a se ta ROS tema odnosi (slika \ref{fig:multirobotexample}). Prefiks je potrebno definirati u glavnoj \emph{launch} datoteci simulacije kao �to je u sljede�em isje�ku XML koda gdje je prefiks string \emph{tb} u \emph{default} svojstvu argumenata. \\

\begin{lstlisting}
<launch>
	<arg name="model" default="$(env TURTLEBOT3_MODEL)" doc="model type [burger, waffle, waffle_pi]"/>
	<arg name="first_tb3"  default="tb1"/>
	<arg name="second_tb3" default="tb2"/>
</launch>
\end{lstlisting}

Ova je funkcija napravljena na na�in da u sceni postoji dodatni neaktivni \emph{RosConnector} i model robota, koji se ovisno o ulazu na glavnom izborniku uklju�uje. Pri pretpla�ivanju pojedina�nog robota na ROS teme, u glavnoj se ROS\# \emph{Unity Subscriber} skripti provjerava broj robota te se u ovom slu�aju dodaje spomenuti prefiks u ROS temu. Promjena robota (kamere) se vr�i na na�in da se uklju�uje kamera jednog robota a isklju�uje kamera drugog robota a obe kamere su namje�tene kao glavni prikaz.

Pri pokretanju i spajanju na simulaciju, prija�nje navedenim gumbom mogu�e je mijenjati pogled s jednog robota na drugi. U ovoj se simulaciji roboti kre�u sami te su ostale funkcije onemogu�ene.

\begin{figure}[!htbp]
	\begin{center}
		\includegraphics[height=8cm,width=16cm,keepaspectratio=true]{multiRobotExample}
		\caption{Primjer ROS tema s vi�e robota}
		\label{fig:multirobotexample}
	\end{center}
\end{figure}

\clearpage

\subsection{3D mapiranje}

\qquad Za izradu podataka 3D mape kori�ten je \emph{octomap\_mapping} koji izra�uje 3D mre�u popunjenosti (eng. 3D occupancy grid). Navedena metoda se koristi pokretaju�i \emph{octomap\_server} �vor koji omogu�ava inkrementalno izgra�ivanje i spremanje mape te distribuira mapu ostalim �vorovima preko svojih ROS tema. \cite{octomap}

\qquad Analizom dostupnih \emph{octomap} �vorova odlu�eno je da bi najprikladniji �vor za Unity bio \emph{/octomap\_point\_cloud\_centers} koji �alje poruke formata \emph{sensor\_msgs/PointCloud2}. PointCloud2 je prikladan format za Unity iz razloga �to sadr�i kolekciju N-dimenzionalnih to�aka, u tom slu�aju 3 dimenzije (x, y, z). Navedeni �vor objavljuje to�ke u stvorenoj 3D mapi napravljenoj od strane \emph{octomap} alata. Naime, odabrani �vor je potrebno procesirati da bi se iz njega dobilo x, y, z to�ke. U tom cilju potrebno je napraviti dodatni �vor koji se pretpla�uje na odabrani �vor, procesira podatke te ih procesirane objavljuje u novi �vor na kojega �e se Unity pretpla�ivati. Za procesiranje podataka kori�tena je gotova funkcija za tu svrhu iz \emph{ros\_numpy} knji�nice koja adaptira poznati Python-ov \emph{numpy} za ROS, \emph{point\_cloud2.pointcloud2\_to\_array}, no dobiveno polje podataka nije mogu�e objaviti bez da postoji ROS poruka tog formata. U tu svrhu napravljene su dvije nove ROS poruke, \emph{CustomPointCloud} i \emph{CustomPointCloudMsg}.

\emph{CustomPointCloud} sadr�i to�ke x, y, z u float32 formatu dok \emph{CustomPointCloudMsg} sadr�i polje \emph{CustomPointCloud} poruka. Kombiniranjem dvije poruke jedini je na�in za napraviti intuitivni format podataka za lak�e naknadno procesiranje u Unity-ju.

Imaju�i na raspolaganju prilago�ene ROS poruke za potrebe ovog rada sada je mogu�e u prija�nje navedenom novom �voru napraviti novu poruku formata \emph{CustomPointCloudMsg} u kojem �e se spremiti sve to�ke u 3-dimenzionalnom prostoru te objavljivati na novi �vor \emph{/octomap\_point\_cloud\_centers\_filtered}.

Sljede�e je potrebno izmijeniti ROS\# knji�nicu, tj. dodati nove ROS poruke jer ina�e Unity nebi mogao primati poruke stvorenog �vora. U ROS\# .NET projekt dodaju se dvije nove klase u mapu \emph{MessageTypes} koje implementiraju isti format podataka kao stvorene ROS poruke. Projekt treba ponovno izgraditi te izmijenjene .dll datoteke dodati u Unity projekt.

U aplikaciji se vr�i klasi�nu pretplatu na novi �vor \emph{/octomap\_point\_cloud\_centers\_filtered} te dobivene poruke spremamo u novostvoreni format. Stvaranje i prikaz 3D mape se vr�i na sljede�i na�in:

\begin{enumerate}
	\item Pri primitku nove poruke, pokre�e se funkcija za stvaranje objekata u sceni koji �e reprezentirati dobivene to�ke 3D mape.
	\item Petljom prolazimo kroz sve dobivene to�ke. Za svaku to�ku stvara se kocku kojoj se dodijeljuje veli�ina i x, y, z pozicija u Unity prostoru.
	\item Primitkom nove poruke, ponovno se pokre�e funkcija za stvaranje objekata (a�uriranje mape) samo ako je prija�nje pozivanje funkcije gotovo.
\end{enumerate}
Ovim koracima stvara se 3D mapa u Unity aplikaciji.

\subsection{RQT graf sustava}

\qquad Na slici \ref{fig:rqtgraph} prikazan je cjelovit RQT graf sustava. \emph{rqt\_graph} je grafi�ki alat za prikaz ROS grafa pokrenutog sustava/aplikacije. U njemu su prikazane poveznice izme�u pokrenutih ROS tema i �vorova.

U grafu je vidljivo da sve ROS teme s povratnim informacijama idu u \emph{/ros\_websocket} a iz njega odlaze poruke u ROS teme za upravljanje robotom.

\begin{figure}[!htbp]
	\begin{center}
		\includegraphics[height=8cm,width=16cm,keepaspectratio=true]{rqtGraph}
		\caption{RQT graf sustava}
		\label{fig:rqtgraph}
	\end{center}
\end{figure}

 
% !TeX encoding = windows-1250
\chapter{Rezultati}

\section{Aplikacija}

\qquad Uspje�no su napravljene funkcionalne aplikacije za sve �eljene platforme (Windows, Linux i Android).

\subsection{Windows i Linux}

\qquad Ove verzije aplikacije su iste po izgledu i funkcionalnostima pa �e se zato zajedno analizirati.

\subsection{Android} 
% !TeX encoding = windows-1250
\chapter{Zaklju�ak}

\qquad U ovom se radu razvila aplikacija za upravljanje robotom i mapiranje prostora koriste�i SLAM gmapping. Paralelno s upravljanjem robota generira se 2D mapa prostora, nalik tlocrtu zgrade. Tako�er, omogu�eno je generiranje sfera iz podataka laserskog skena, kao i prikaz iz prvog lica kamere robota. U slu�aju vi�e robota u simulaciji, mogu�a je promjena perspektive s jedne kamere robota na kameru drugog robota.

Aplikacija je ra�ena isklju�ivo koriste�i Turtlebot3 simulaciju. Za adaptaciju na drugu vrstu robota trebalo bi biti dovoljno u�itati novi URDF model te eventualno izmijeniti ROS teme u \emph{RosConnector-u} u slu�aju da se razlikuju.

Aplikacija je izgra�ena i testirana na tri platforme: Windows, Linux Ubuntu i Android. Mogu�e je uz minimalne promjene aplikaciju izgraditi i za Mac OS. Performanse aplikacija nisu savr�ene kao �to je navedeno u zadnjem poglavlju rada, no aplikacija za Windows je drasti�no lo�ija od ostalih iako je sve prakti�ki isti kod.

ROS\# se pokazao kao alat s vrlo velikim potencijalom za izradu konkretnijih aplikacija, kao i za edukacijske svrhe kao te za entuzijaste koji rade s Unity razvojnim okru�enjem.

Zaklju�no, glavni cilj ovog rada - spajanje robota na Unity i implementiranje multiplatformske aplikacije, dodatno podi�e mogu�nosti ROS-a i Unity-ja koji zajedno mogu dovesti do vrlo zanimljivih i korisnih rezultata. 
\include{Literatura}  % ovo je ime Bibtex datoteke koju korisnik kreira


%:::::::::: ukljucenje popisa kratica u tekst ::::::::
% Blok linija koda ispod ovoga generira ukljucenje popisa kratica u tekstu. Za uporabu, vidjeti Upute.
% Nije obavezno. Ako se ne zeli koristiti, onda ovaj blok staviti u komentar pomocu znaka %
%\printglossary[type=\acronymtype]
%\pagestyle{plain}
%\begin{glossary}{Longest string}%
%	\input{Kratice}
%\end{glossary}


%:::::::::::: blok za definiranje Sazetka/Abstracta rada 
\begin{abstract}
	% !TeX encoding = windows-1250
\vspace{5pt}

%:::::::::::::::::::::::::::::::::::::::::::::::::::::
%:::::::::::: HRVATSKI :::::::::::::::::::::::::::::::
\noindent
Cilj ovog rada bio je napraviti funkcionalnu aplikaciju za upravljanje robotom i mapiranje prostora koriste�i Unity razvojno okru�enje. Za spajanje ROS sustava i Unity razvojnog okru�enja kori�ten je ROS\# alat. Kartiranje se vr�i paralelno uz upravljanje robotom. Izra�uje se 2D karta koja se prikazuje ispod robota. Omogu�en je prikaz podataka s laserskog skena u obliku sfera i prikaz iz prvog lica kamere robota. U svrhu istra�ivanja napravljeno je da se u slu�aju dvaju robota u simulaciji mo�e mijenjati pogled kamere s jednog robota na drugi. Implementirano je i generiranje 3D karte, ali u zasebnoj sceni. Aplikacija je funkcionalna na tri platforme: Windows, Linux i Android. 
%:::::::::::::::::::::::::::::::::::::::::::::::::::::

\vspace{5pt}
%
\noindent \textbf{\textit{Klju�ne rije�i} --- ROS, Unity, ROS\#, upravljanje robotom, kartiranje} 

%:::::::::::: KRAJ HRVATSKOG DIJELA :::::::::::::::::::


%::::::::::::::::::::::::::::::::::::::::::::::::::::::
%:::::::::::: ENGLESKI ::::::::::::::::::::::::::::::::

%\vspace{-10pt}
\section*{Abstract}
\vspace{-10pt}
The objective of this thesis was to make a functional application for robot control and mapping using the Unity game engine. The connection of ROS and Unity game engine has been made with the ROS\# tool. Mapping is done in parallel with the robot control. A 2D map is being made which is shown under the robot. Laser scan data is displayed in a form of spheres and a first person view from the robot camera was also enabled. For research purposes in the case of two robots in the same simulation, a feature was created in order to switch the camera view from one robot to another. 3D mapping is also implemented, but on a separate scene. The application is functional on three platforms: Windows, Linux and Android. 
%:::::::::::::::::::::::::::::::::::::::::::::::::::::::

\vspace{5pt}
%
\noindent \textbf{\textit{Keywords} --- ROS, Unity, ROS\#, robot control, mapping}

%::::::::::::::::::::::::::::::::::::::::::::::::::::::
%:::::::::::: KRAJ ENGLESKOG DIJELA :::::::::::::::::::

	  % sazetak rada i kljucne rijeci na HR i EN
\end{abstract}


%:::::::::::: PRILOZI (neobavezno) ::::::::::::::::::::
% ispod \appendix zaglavlja pomocu \include dodati poglavlja s prilozima
% ukoliko nemate priloga, ovaj blok linija staviti u komentar
\appendix
% !TeX encoding = windows-1250
\chapter{Konfiguracija radne okoline}
\qquad U ovom �e se poglavlju opisati postupak konfiguriranja radne okoline. Ovo je potrebno iz razloga �to ve�ina alata nije testirana na najnovijim Ubuntu i ROS verzijama pa su odre�ene adaptacije bile potrebne.

\qquad Instalaciju i konfiguraciju se mo�e podijeliti u nekoliko koraka, od kojih �e se pojasniti samo one koji su zahtjevali dodatne korake:
\begin{enumerate}
	\item Instalacija Ubuntu 20.04 OS-a.
	\item Instalacija ROS-a Noetic Ninjemys-a gdje je potrebno pratiti upute na slu�benim stranicama. Vr�i se instalacija potpunog paketa.
	\item Instalacija Unity-ja. Dobro je povremeno instalirati noviju verziju jer sadr�i korisna a�uriranja.
	\item Instalacija Visual Studio Code-a - text editor koji �e se koristiti uz Unity za programiranje skripta. Razlog odabira Visual Studio Code-a je taj �to je potrebno imati podr�ku za Unity i mogu�nost spajanja i pokretanja alata za otklanjanja pogre�ka (debugger).
	\item Instalacija .NET Core radne okoline za omogu�iti rad Unity-ja s Visual Studio Code - podr�ka za C\#. \cite{netcoreinstall}
	\item Visual Studio Code konfiguracija za Unity. \cite{vscodeconfig} 
	\item Instalacija \textit{Mono} radne okoline. \cite{monoinstall} Preporu�ljivo je nakon ovog koraka pokrenuti i naredbu \textit{sudo apt install mono-complete} da bi se instalirali eventualni segmenti koji fale.
	\item Instalacija Turtlebot 3 paketa za ROS 1. \cite{turtlebotinstall}
\end{enumerate}

\section{Dodatni koraci}
\qquad Ovdje �e biti navedeni i eventualno poja�njeni dodatni koraci. Neki od tih su jednostavna instalacija dodatnog paketa, a neki promjene da bi odre�ena stvar mogla proraditi.

\qquad Kada je neki alat ili knji�nicu u ROS-u potrebno instalirati iz izvornog koda, to se u osnovi radi na sljede�i na�in (eventualni alati imaju specificirana potrebna dodatna pode�avanja):
\begin{enumerate}
	\item Preuzimanje i spremanje mape s datotekama izvornog koda u ROS radni folder (workspace - \textit{~/catkin\_ws/src/}) - ova mapa je proizvoljna ali ROS standard je \textit{catkin\_ws} u \textit{home} direktoriju Linux Ubuntu OS-a.
	\item U korijenskom \textit{catkin\_ws} direktoriju pokre�e se naredba \textit{catkin\_make} koja izgradi sav kod u njemu. Tada se pojave dvije nove mape - \textit{build} i \textit{devel}. Nakon toga treba pozvati izgenerirani \textit{setup.bash} sljede�om naredbom koja omogu�ava kori�tenje novo izgra�enih paketa:
	
\textit{source /home/user/catkin\_ws/deve/setup.bash} ili \textit{source /opt/ros/noetic/setup.bash}
\end{enumerate}

\begin{itemize}
	\item Za omogu�iti mapiranje robotske okoline potrebno je dodatni instalirati \textit{slam} pakete za mapiranje. Instalacija se vr�i iz izvornog koda kojeg je mogu�e dohvatiti s git repozitorija paketa \cite{percgit} gdje se nalazi i drugih paketa za robotsku percepciju. 
	\item Osnovna spona koja omogu�uje slanje podataka izme�u ROS-a i Unity-ja je \textit{RosBridge} paket kojeg ina�e dohva�amo naredbom \textit{sudo apt-get install ros-noetic-rosbridge-server}. Istoga koristi ROS\#. ROS Noetic-u instalacija ovog paketa na klasi�an na�in ne radi, ali vi�e o tome u sljede�em odlomku.
	\item Za kori�tenje �eljenog robota u Unity, prvi korak je unijeti njegov model. Koristi se URDF (Universal Robotic Description Format - univerzalni robotski opisni format). URDF je pisan u XML formatu i koristi se za opisati sve elemente opisanog robota. 
\end{itemize}

\subsection{Uvoz URDF modela u Unity}
\qquad Za uvoz URDF modela potrebno je pokrenuti odre�enu \textit{launch} datoteku koja se nalazi u ros-sharp (datote�no prihvatljiv naziv za ROS\#) paketu. Datoteka se nalazi u \textit{ros-sharp/ROS/file-server/launch} direktoriju koja se pokre�e \textit{roslaunch publish\_description\_turtlebot.launch} no prije samog pokretanja potrebno ju je urediti iz razloga �to je to samo primjer za Turtlebot 2 robot, koji ima druga�ije specifikacije od Turtlebot 3 robota.

\qquad \textit{Launch} datoteke su tako�er pisane u XML formatu i one slu�e za pokretanje vi�e �vorova odjednom. U datoteci se mo�e podesiti i dodatne parametre za pokretanje odre�enih �vorova. \textit{Roslaunch} je kori�ten za pokretanje tih datoteka i to se mo�e u�initi pozivanjem paketa i specifi�ne \textit{launch} datoteke ili direktno pozivanjem \textit{launch} datoteke definiranjem njezine datote�ne putanje:

\qquad - \textit{roslaunch ime\_paketa launch\_datoteka}

\qquad - \textit{roslaunch ../catkin\_ws/src/paket/launch/launch\_datoteka}

\qquad Osim ure�ivanja robotskih vrijednosti, potrebno je i urediti i argument \textit{urdf\_file} na na�in da se iz \textit{xacro.py} izbaci \textit{.py} te doda flag \textit{--inorder}. To se mora iz razloga �to su sve prija�nje verzije ROS-a bile na Python 2, ali je ROS Noetic na Python-u 3.

\qquad Jo� jedna promjena koja je potrebna a povezana je s verzijama Python-a je za uspje�no pokretanje \textit{RosBridge} poslu�itelja. Naime, ako se izvr�i instalacija standardnom \textit{apt install} naredbom, RosBridge ne�e raditi jer se kose vrste varijabla (\textit{str} i \textit{byte}) koje su druga�ije definirane u svakoj verziji Python-a, pa je iz tog razloga potrebno instalirati cijeli RosBridge iz izvornog koda \cite{rosbridgegit} uz namje�tenu to�nu verziju Python-a (2).

\qquad Nakon ovih koraka mo�e se pokrenuti Unity i u izborniku odabrati \textit{RosBridgeClient -> Transfer URDF from ROS} gdje se otvara prozor�i� (slika~\ref{fig:urdfimport}) u kojemu je potrebno izmijeniti IP adresu gdje �e se prona�i RosBridge poslu�itelj. Ostale vrijednosti bi trebale biti dobre po zadanim vrijednostima. Prije spajanja, potrebno je pokrenuti TurtleBot simulaciju i gore navedenu \textit{launch} skriptu. Dovr�etkom ovog koraka, u Unity projekt se sprema URDF model kojega se mo�e prikazati i koristiti u scenama.

\begin{figure}[!htbp]
	\begin{center}
		\includegraphics[height=6cm,width=12cm,keepaspectratio=true]{URDFImport}
		\caption{URDF uvoz}
		\label{fig:urdfimport}
	\end{center}
\end{figure}


\qquad Za kori�tenje simuliranog okru�enja koristi ste \textit{Gazebo} softver, no iz razloga limitirane kompatibilnosti umjesto zadnjeg Gazebo 11, koristi se Gazebo 9. nakon brisanja Gazebo 11, Gazebo 9 se instalalira sljede�im naredbama, zajedno s potrebnim paketima:

\qquad \textit{sudo apt install gazebo9-common}

\qquad \textit{sudo apt-get install libgazebo9-*}

\qquad \textit{sudo apt install ros-noetic-gazebo-ros-pkgs}



% !TeX encoding = windows-1250

\section{Glavna launch datoteka}

\begin{lstlisting}
<launch>
	<include file="$(find rosbridge_server)/launch/rosbridge_websocket.launch">
	<param name="port" value="9090"/>
	</include>
	<include file="$(find turtlebot3_slam)/launch/turtlebot3_slam.launch">
	<param name="slam_methods" value="gmapping"/>
	</include>
	
	<!-- One robot -->
	<include file="$(find turtlebot3_gazebo)/launch/turtlebot3_house.launch">
	</include> 
	
	<!-- Multiple robots 	
	<include file="$(find turtlebot3_gazebo)/launch/multi_turtlebot3.launch">
	</include> -->	
	
	<node name="file_server" pkg="file_server" type="file_server" />
	
	<node name="joy_to_twist" pkg="gazebo_simulation_scene" type="joy_to_twist.py"/>
	
	<node name="rqt_graph" pkg="rqt_graph" type="rqt_graph" />
	
	<node name="laserScanCorrector" pkg="gazebo_simulation_scene" type="laserScanCorrector.py" />
</launch>
\end{lstlisting} 
% !TeX encoding = windows-1250

\section{Ostalo}

\qquad Ostale relevantne launch datoteke, Unity projekt i drugo biti �e prilo�eni kao zasebne datoteke.
  



\end{document}
