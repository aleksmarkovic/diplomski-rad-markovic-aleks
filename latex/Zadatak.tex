% !TeX encoding = windows-1250
\chapter{Uvod}
\qquad Tema ovog diplomskog rada je napraviti funkcionalnu aplikaciju za upravljanje robotom i mapiranje okoline koriste�i Unity 3D razvojni program. Zahvaljuju�i Unity-ju biti �e lak�e ostvariti cilj da se napravi univerzalni i multiplatformski softver s kojim �e se mo�i upravljati s vi�e vrsta robota.

Po�to je danas robotika prisutna u skoro svakom segmentu na�ega �ivota, sve su popularnije i potrebnije aplikacije za upravljanje i interakciju sa robotima �to daje dodatnu motivaciju za ovaj rad.

Kao glavni alat za spajanje i upravljanje na robota koristi se ROS 1 (Robotski Operacijski Sustav).
Za omogu�avanje komunikacije izme�u ROS-a, tj. robota i Unity aplikacije, koristiti �emo ROS\# knji�nicu.
Za svrhu implementacije i testiranja kao testnog robota odabran je popularni Turtlebot 3. Konkretnije koristit �emo simulirano okru�enje (simulaciju) Turtlebot-a i njegovog modela.

ROS\# je relativno novi alat, s prvom verzijom razvijenom po�etkom 2018. godine. Stoga nema previ�e podr�ke diljem interneta �to �ini ovaj rad zanimljivijim i smislenijim.

Rad je podijeljen u 4 glavna dijelova, po�ev�i od opisa problema pa do analize kori�tenih programskih alata te do implementiranog rje�enja i njegovih rezultata.

\clearpage