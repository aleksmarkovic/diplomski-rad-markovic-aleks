% !TeX encoding = windows-1250
\chapter{Softverski alati}
\qquad Prije samog rije�avanja problematike kako napraviti navedenu aplikaciju, potrebno je objasniti ?to su i kako funkcioniraju kori?teni softverski alati. Definirati ?e se i koji su preduvjeti, tj. knji?nice ili alati koji svaki od njih zahtjeva da se mo?e odraditi funkcija koja im se zada za prethodno navedenu svrhu.

\section{ROS}
\qquad Robotski Operacijski Sustav (ROS) je radni okvir (eng. framework) koji se instalira u Linux operacijski sustav. Iako sadr�i rije�i operacijski sustav, on to nije. Postoji i eksperimentalna verzija za Windows 10 i OS X, no ovaj �e se rad usredoto�iti na razvoj na Linux-u. Jedna od najbitnijih karakteristika ROS sustava jest da je omogu�ena komunikacija i upravljanje hardverom robota preko softverskih alata ROS-a bez da se treba imati posebno znanje o kori�tenom hardveru.
\qquad ROS se ponajvi�e koristio u znanstvene i obrazovne svrhe, ali se zbog svoje prakti�nosti i potencijala ubrzo pro�irio i u ostale grane robotike. Prije prelaska na ROS, svaki proizvo�a� robota je je trebao razvijati svoj API (Application Programming Interface) za komunikaciju i upravljanje svojim robotima. Sada roboti diljem svijeta ve�inom koriste ROS kao svoj primarni sustav za komunikaciju i upravljanje, te je zbog toga vrlo korisno nau�iti ROS. Sa istim znanjem i vje�tinama mogu�e je razvijati softver koji �e poslu�iti na razli�itim robotima, razli�itih proizvo�a�a, upravo radi ROS unificiranja. 
\qquad ROS sadr�i razne alate i knji�nice, koji su razvijeni i poslo�eni po odre�enoj ROS konvenciji. Sve zajedno jako pojednostavljuje razvoj novih robotskih softvera i omogu�ava kompleksno pona�anje robota. \emph{rosurl}

\section{Unity 3D}
\section{ROS\#}

% itd.

%%%%  POGLAVLJE ZAVRSENO  %%%%%
