% !TeX encoding = windows-1250
\chapter{Uvod}
\qquad Robotika kao znanost moderne sada�njice, sveprisutna je u svakom segmentu na�eg �ivota, �to ukazuje potrebu i nu�nost izrade aplikacija za upravljanje i interakciju sa robotima. 

Cilj ovog diplomskog rada je napraviti funkcionalnu aplikaciju za upravljanje robotima i mapiranje okoline koriste�i Unity razvojno okru�enje. Unity �e znatno olak�ati izradu univerzalnog i multiplatformskog softvera za upravljanje vi�e vrsta robota.

Kao glavni alat za spajanje i upravljanje na robota koristi se ROS 1 (Robotski Operacijski  Sustav) (dalje u tekstu: ROS), a da bi se omogu�ila komunikacije  izme�u ROSa,  tj.   robota  i Unity-ja, koristiti �e se ROS\# knji�nicu. Implementacija i testiranje robota  provest �e se uz  popularni Turtlebot  3, za �to �e se koristiti simulirano okru�enje, odnosno simulacija Turtlebota i njegovog modela.

S obzirom da je ROS\# noviji alat, s prvom verzijom razvijenom po�etkom 2018. godine, nema previ�e podr�ke diljem interneta �to �ini ovaj rad zanimljivijim, smislenijim i izazovnijim.

Rad je razra�en u 5 poglavlja, od kojih po�etno poglavlje opisuje problematiku i potencijalne izazove ovog rada. Sljede�a dva poglavlja opisuju �to se u radu koristilo od tehnologija, od specifikacije robota i njegovih komponenta koje su bile potrebne za omogu�avanje odre�enih zna�ajki, pa do programskih alata koje se koristilo za implementaciju cjelokupnog rje�enja. Zadnja dva poglavlja usko su vezana za sam razvoj rje�enja. Poglavlje opisa rje�enja navodi sve implementirane mogu�nosti i kako iste funkcioniraju, dok poglavlje rezultata prikazuje i obja�njava krajnji rezultat razvoja (aplikaciju) te navodi probleme i izazove do kojih se nai�lo u cjelokupnom procesu razvoja.

\clearpage