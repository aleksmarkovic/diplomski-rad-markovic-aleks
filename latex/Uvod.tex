% !TeX encoding = windows-1250
\chapter{Uvod}
\qquad Robotika kao znanost moderne sada�njice, sveprisutna je u svakom segmentu na�eg �ivota, �to ukazuje potrebu i nu�nost izrade aplikacija za upravljanje i interakciju sa robotima. 

S obzirom da je cilj ovog diplomskog rada napraviti funkcionalnu aplikaciju za upravljanje robotima i mapiranje okoline, za izradu iste koristit �e se Unity razvojno okru�enje s kojim �e se jednostavnije ostvariti postavljeni cilj izrade univerzalnog i multiplatformskog softvera za upravljanje vi�e vrsta robota.  

Kao glavni alat za spajanje i upravljanje na robota koristi se ROS 1 (Robotski Operacijski  Sustav) (dalje u tekstu: ROS), a da bi se omogu�ila komunikacije  izme�u ROS-a,  tj.   robota  i Unity-ja, koristiti �e se ROS\# knji�nicu. Implementacija i testiranje robota  provest �e se uz  popularni Turtlebot  3, za �to �e se koristiti simulirano okru�enje, odnosno simulacija Turtlebot-a i njegovog modela.

S obzirom da je ROS\# noviji alat, s prvom verzijom razvijenom po�etkom 2018. godine, nema previ�e podr�ke diljem interneta �to �ini ovaj rad zanimljivijim, smislenijim i izazovnijim.

Rad je razra�en u 4 glavna dijela, po�ev�i s opisom problema pa do analize kori�tenja hardverskih i programskih alata te do implementiranog rje�enja i njegovih rezultata.

\clearpage